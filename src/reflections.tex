\chapter{Reflections} \label{cha:reflections}
The first part of this thesis presented the background and five examples for the design of visualization applications that utilize a combination of scientific visualization, information visualization, and visual analytics techniques.

This final chapter provides a summary of this first part of the thesis, includes personal reflections and uses these as possible venues for future work.  The future work is both personally as well as humble suggestions for the visualization community as a whole.  In depth discussions for concrete potential future work in each domain discussed in the previous chapted can be found in the publications in the second part of this thesis.

This thesis follows the line of argumentation that emphasizes the importance of considering the intended target audience while designing and creating any visualization application.  Building on previous realizations on the difference between applications designed for \emph{explorational} and \emph{presentational} use cases by van Wijk~\cite{van2005value} and Keim~\cite{keim2006challenges}, the papers selected for inclusion in this thesis illustrate that the explorational use cases presented by van Wijk can be further subdivided into initial exploratory hypothesis generation and repeated analyses.  This leads to the three categories of visualization applications, \emph{Exploration}, \emph{Analysis}, and \emph{Communication} described in Chapter~\ref{cha:intro}.

While there has always been exchanges between visualization applications designed for exploration and analysis, in the past the influence of these two categories on the communication aspect of visualization was less pronounced.  The currently ongoing development of the visualization application portrayed in Section~\ref{contributions:astro:openspace} shows the potential of utilizing guided exploration for the public dissemination of complex scientific topics and how simplified exploratory methods can be used for the education of the general public.

Undoubtedly, the science of visualization and its tools have matured a lot in the past 25 years and now has a large influence on the scientific method in almost every field of research.  High-impact scientific publications without accurate accompanying visualizations of their findings are unthinkable today.  Furthermore, in many cases, no visualization experts are necessary to create these final visualizations as the access to sophisticated visualization tools has increased dramatically.  The increasing ease-of-use of existing visualization techniques has thus led to a proliferation of knowledge about visualization techniques.  While this, in my opinion, is a worthwhile trajectory of visualization research as it enables access to these tools to an increasing number of people, it also critically important to increase the users' visualization literacy alongside the tools' ease-of-use.

The trend on focussing on the accessibility of visualization methods for the general public was present since the beginning of the visualization field, but has been increasing in the past years.  The availability of visualization tools is especially beneficial for an interested general public.  Providing access to sophisticated tools for the general public can provide a huge benefit to citizen science projects which increases the combined human knowledge and scientific engangement simultaneously.  Naturally, this does not argue for the stagnation of fundamental visualization research but rather argues for an increase in the number of visualization researchers (another trend which can be observed at conferences).

In my opinion, one of the most interesting and important developments that are currently ongoing is the intersection of visualization and machine learning methods.  One of the most basic concepts of visualization is to connect the best abilities of the human mind with the best abilities of computational resources.  Combining the natural intelligence of a human expert with the artificial intelligence of deep learning methods can be hugely beneficial for the field.  While, there are interesting debates on the viability of artificial intelligence to replace human cognition, these are better left outside the scope of this work.  But until/if that happens, visualization techniques can certainly benefit from the rapid development in deep learning and vice versa.  In the reference frame of visualization as the field of using the human intelligence where it performs best and computational resources where they are most efficient, the introduction of artificial intelligence merely shifts that boundary, rather than removing it completely.  This is illustrated by the quote from Stefan Lindholm: ``One trend is the ever increasing number of arrows used whenever a visualization pipeline is illustrated. The narrow concept of a linear pipeline with a few fixed stages is going extinct''~\cite{lindholm14medical}.  A suitable addition to this quote is the realization that it is important to remember that in the visualization pipeline, all arrows start and end at the human-in-the-loop.
