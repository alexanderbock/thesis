\chapter{Reflections} \label{cha:reflections}

This final chapter provides a summary of this first part of the thesis, includes personal reflections and uses these as possible venues for future work, both personally as well as for the visualization community itself.  Detailed discussions of future work for each domain discussed in the previous chapter can be found in the papers that constitute the second part of the thesis.

This thesis makes an argument for the consideration of the intended target audience when designing a visualization application.  Building on previous realizations on the difference between applications designed for \emph{explorational} and \emph{presentational} use cases by van Wijk~\cite{van2005value}, the papers selected for this thesis show that the explorational use cases can be further subdivided into pure exploratory hypothesis generation and repeated (re-)production usages .  This leads to the three categories of visualization applications, \emph{Exploration}, \emph{Production}, and \emph{Public Dissemination}.  The ongoing development of the application portreyed in the last example in Chapter~\ref{cha:contributions} shows the potential of utilizing guided exploration for the public dissemination of complex scientific topics and how simplified exploratory methods can be used for the education of the general public.

Undoubtedly, the sciene of visualization and its tools have matured a lot in the past 25 years and now has a great influence on the scientific method in almost every imaginable field.  High-impact scientific publications without accurate visualizations of their findings are unthinkable today.  Furthermore, in many cases, no visualization experts were necessary to create these final visualization as available tools now provide easy access to sophisticated visualization techniques.  In my opinion, this trajectory of increasing the ease-of-use of existing visualization technologies and the subsequent proliferation of knowledge of visualization techniques should exactly be the desired trajectory of visualization research a enables an ever-greater number of people access to these tools.  The trend on focussing on the accessibility of tools for the general public was present since the beginning of the visualization field, but is increasing dramatically in the past years.  The availability of visualization tools is especially beneficial for an interested general public.  Providing access to sophisticated tools for the general public can provide a huge benefit to citizen science projects which increases the combined human knowledge and scientific engangement simultaneously.

Naturally, this does not argue for the stagnation of pure visualization research but rather argues for an increase in the number of visualization researchers (another trend which can be observed at major conferences).  In my opinion, one of the most interesting and important developments that are currently ongoing is the intersection of visualization and machine learning methods.  Combining the natural intelligence of a human expert with the artificial intelligence of deep learning methods can be hugely beneficial for the field.  While, there are interesting debates on the viability of artificial intelligence to replace human cognition, these are better left outside the scope of this work.  But until/if that happens, visualization techniques can certainly benefit from the rapid development in deep learning and vice versa.  In the reference frame of visualization as the field of using the human intelligence where it performs best and computational resources where they are most efficient, the introduction of artificial intelligence merely shifts that boundary, rather than removing it completely.  This is illustrated by the quote from Stefan Lindholm: ``One trend is the ever increasing number of arrows used whenever a visualization pipeline is illustrated. The narrow concept of a linear pipeline with a few fixed stages is going extinct.''~\cite{lindholm14medical}  In this frame, the increased use of machine learning methods is only yet another arrow that can be added to the visualization pipeline since in the end, all that visualization revolves around is the human mind.

% In no area is this continuing advance more obvious than the general public.  While there are still many advancements to be made, the increased exposure to visualization and interaction techniques 



% Even though development in the field of software engineering has stagnated for a long time, there are still some methodologies that can be adapted to be used in visualization application design.  

% Getting solutions for future applications
% What things would make future application design easier
% Influence on the title

% Connection to software engineering (learning from software engineering to a degree)
% Visualization have had a great influence on the scientific method (domain papers without illustrations unthinkable today)
% Visualization has matured in the past 25 years
% Heaver focus on the dissemination of data to the public than before (as tools mature, this becomes easier;  home computers getting powerful enough)
% Move into time-varying data

% Increasing knowledge encoding
% Stefan: "One trend is the ever increasing number of arrows used whenever a visualization pipeline is illustrated. The narrow concept of a linear pipeline with a few fixed stages is going extinct."

% Symbiosis with artificial intelligence
% Augmenting the visualization pipeline (natural intelligence) with artificial intelligence
