\chapter*{Publications}
\addcontentsline{toc}{chapter}{List of publications}
\pdfbookmark[0]{Publications}{publications}

The following list of publications have been included in this thesis:
\begin{description}[leftmargin=!,labelwidth=\widthof{\bfseries Paper X:}]
% \itemsep18pt
\item[{\hyperref[pa:paperA]{Paper A:}}] \bibentry{bock12coherency}
\item[{\hyperref[pa:paperB]{Paper B:}}] \bibentry{bock13guiding}
\item[{\hyperref[pa:paperC]{Paper C:}}] \bibentry{bock14supporting}
\item[{\hyperref[pa:paperD]{Paper D:}}] \bibentry{bock14interactive}
\item[{\hyperref[pa:paperE]{Paper E:}}] \bibentry{bock16visualization}
\item[{\hyperref[pa:paperF]{Paper F:}}] \bibentry{bock15visual}
\item[{\hyperref[pa:paperG]{Paper G:}}] \bibentry{bock17dynamic}
\item[{\hyperref[pa:paperH]{Paper H:}}] \bibentry{bladin17globe}
\item[{\hyperref[pa:paperI]{Paper I:}}] \bibentry{bock18openspace}
\end{description}

\newcommand{\paperFEM}{\paperef{paperA}}
\newcommand{\paperDBS}{\paperef{paperB}}
\newcommand{\paperVMV}{\paperef{paperC}}
\newcommand{\paperSSRR}{\paperef{paperD}}
\newcommand{\paperCGF}{\paperef{paperE}}
\newcommand{\paperCME}{\paperef{paperF}}
\newcommand{\paperDSG}{\paperef{paperG}}
\newcommand{\paperGB}{\paperef{paperH}}
\newcommand{\paperOS}{\paperef{paperI}}

\newpage

The following publications, reported in reverse chronological order, are related to the work described in this thesis, but have not been included:

\begin{itemize}
    \item \bibentry{bock17openspace}
    \item \bibentry{bock15bopenspace}
    \item \bibentry{bock15openspace}
    \item \bibentry{dieckmann15shocks}
    \item \bibentry{bock14vcmass}
    \item \bibentry{sunden14interaction}
    \item \bibentry{lindholm14hybrid}
    \item \bibentry{lindholm13poor}
    \item \bibentry{nguyen12deriving}
    \item \bibentry{liu12gpu}
\end{itemize}

\clearpg

\chapter*{Contributions}
\addcontentsline{toc}{chapter}{Contributions}
\pdfbookmark[0]{Contributions}{contributions}

\newcommand{\absubsubsectionlength}{0mm}
% \newcommand{\absubsubsectionlength}{-2.5mm}

\subsubsection{Paper A:~~Coherency-Based Curve Compression for High-Order Finite Element Visualization}
\vspace*{\absubsubsectionlength}
Presents a novel rendering technique for real-time visualization of non-linar finite element models by introducing a preprocessing step in which potential rays are precomputed by solving non-linear transformations and then compressed through the use of B-splines.  During ray marching, these proxy rays are used as an approximation as the transformations are not achievable in real-time.  The method leads to a performance improvement of 15$\times$ compared to straight-forward GPU implementations. This work was presented at IEEE VisWeek 2012.

\subsubsection{Paper B:~~Guiding Deep Brain Stimulation Interventions by Fusing Multimodal Uncertainty Regions}
\vspace*{\absubsubsectionlength}
In a participatory design with expert brain surgeons, this work presents a system that supports Deep Brain Stimulation operations placing an electrode in the patient's subthalamic nucleus.  The presented system uses the available modalities, such as preoperative CT/MRI scans, interoperative X-ray, probe measurements, and patient responses, and fuses the available information into a multi-view system that presents the available uncertainty ranges to the surgeon during the operation.  This work was presented at the IEEE Pacific Visualization Symposium 2013.

\subsubsection{Paper C:~~Supporting Urban Search \& Rescue Mission Planning through \\Visualization-Based Analysis}
\vspace*{\absubsubsectionlength}
Presents a decision support system displaying a \nD{3} visualization of point cloud measurements obtained from partially collapsed buildings containing potentially trapped and injured victims.  The system uses these point clouds for a semi-automatic path finding algorithm which suggests paths to an operator who uses combined Scientific and Information Visualization techniques to analyse different path attributes.  This paper describes the results of an online study of this system with nine international expert participants.  This work was presented at the International Symposium of Vision, Modeling, and Visualization in 2014.

\subsubsection{Paper D:~~An Interactive Visualization System for Urban Search \& Rescue Planning}
\vspace*{\absubsubsectionlength}
This work presents an improved version of the decision support system published in Paper C focussed towards a presentation for rescue robots experts.  More relevant aspects of an online user study are presented as well as implementations to be able to visualize the point cloud data, the paths, and derived data in immersive environments.  This work was presented at the International Symposium on Safety, Security, and Rescue Robotics in 2014.

\subsubsection{Paper E:~~A Visualization-Based Analysis System for Urban Search \& Rescue Mission Planning Support}
\vspace*{\absubsubsectionlength}
Based on the findings of Papers C and D, this work includes an adaptive sampling method that replaces the previous brute force sampling of the path search space for improved efficiency.  Additional visualization techniques such as projective texturing and bump mapping are included to convey additional information to the rescuer.  Lastly, the work contains an additional eye-tracking user study with four rescuers.  This work was published in Computer Graphics Forum in 2016.

\subsubsection{Paper F:~~Visual Verification of Space Weather Ensemble Simulations}
\vspace*{\absubsubsectionlength}
Presents a visualization system developed in collaboration with space weather analysts for the use in the investigation of space weather.  The system enables the comparison of in-situ measurements performed by satellites with time-varying volumetric simulations of the solar system.  The system was designed in participatory design with the experts at the Community Coordinated Modeling Center, located at NASA's Goddard Space Flight Center and enabled new discoveries about the structure of coronal mass ejections.  This work was presented at IEEE Vis in 2016.

\subsubsection{Paper G:~~Dynamic Scene Graph: Enabling Scaling, Positioning, and Navigation in the Universe}
\vspace*{\absubsubsectionlength}
By utilizing a dynamic coordinate system origin, the framework described in this work supports the simultaneous rendering of scenes with an extent that is larger than the precision of floating points would otherwise allow.  The paper analyses the precision loss that occurs due to floating point arithmetic and, based on these findings, presents a solution that operates on dynamically traversing a scene graph structure.  This work was presented at EuroVis in 2017.

\subsubsection{Paper H:~~Globe Browsing: Contextualized Spatio-Temporal Planetary Surface Visualization}
\vspace*{\absubsubsectionlength}
This paper presents a system that uses a chunked, level-of-detail rendering techniques for the high-fidelity rendering of planetary surfaces, including static and time-varying imagery data and digital elevation models of Earth, the Moon, Mars, and Pluto.  Using these techniques, it becomes possible to make an extensive library of scientific surface data available to the public in their correct spatial context.  This work was presented at IEEE Vis in 2017.

\subsubsection{Paper I:~~OpenSpace: Changing the Narrative of Public Disseminations in Astronomical Visualization from \emph{What} to \emph{How}}
\vspace*{\absubsubsectionlength}
This work presents the open-source framework OpenSpace which supports the interactive visualization of astronomical data in traditional and immersive environments.  The paper advocates the use of shared, immersive experiences as an efficient medium of science dissemination to the general public and provides an overview of the required techniques to achieve this.  The examples presented in the work include various spacecraft missions, such as New Horizons, Rosetta, and OSIRIS-REx, as well as planetary rendering as described in Paper H, and the space weather visualization as described in Paper F.  This work is accepted for publication in Computer Graphics \& Applications 2018.
