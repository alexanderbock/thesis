\chapter*{Popul\"arvetenskaplig Sammanfattning}
\addcontentsline{toc}{chapter}{Popul\"arvetenskaplig Sammanfattning}

De senaste decennierna har det skett en exponentiell \"okning av tillg\"angliga ber\"akningsresurser vilket har lett till en informationsexplosion inom m\aa nga vetenskapliga omr\aa den.  Samh\"allet har d\"armed g\aa tt fr\aa n att vara informationsfattigt till att ha \"overfl\"od av information.  M\"anniskans f\"orm\aa ga att ta in information \"ar dock relativt of\"or\"anderlig.  Visualisering har h\"ar visat sig vara ett effektivt verktyg f\"or att f\aa\ b\"attre insikt i all information.  \"Aven om visualisering har anv\"ants sedan m\"ansklighetens begynnelse har dess betydelse d\"armed \"okat markant i takt med informationstillv\"axten.  Visualisering, som metod och verktyg, utnyttjar ber\"akningsresurser och den m\"anskliga kognitiva f\"orm\aa gan f\"or att omvandla komplexa data till intuitiva visuella representationer och d\"armed mildra informations\"overfl\"odet.

Mycket av forskningen inom visualisering riktar sig direkt eller indirekt till anv\"andare i en specifik dom\"an, till exempel medicin, biologi, fysik med mera.  Till\"ampad forskning syftar till att skapa visualiseringsapplikationer eller system som l\"oser ett specifikt problem inom en dom\"an.  Genom att till\"ampa allm\"an visualiseringsforskning p\aa\ ett konkret problem skapas m\"ojligheter att j\"amf\"ora och analysera anv\"andbarheten av den forskningen.  Visualiseringsapplikationer kr\"aver att dom\"anexperter \"ar inblandade i en designprocess, som ofta involverar ett iterativt arbetsfl\"ode, f\"or att ta fram deras form och funktion.  Applikationerna kan kategoriseras in i tre olika omr\aa den. \emph{Utforskning}, d\"ar anv\"andarna utf\"or en f\"orsta analys av data.  \emph{Analys}, d\"ar samma teknik anv\"ands upprepade g\aa nger p\aa\ m\aa nga olika dataset.  \emph{Kommunikation}, d\"ar resultaten visas f\"or en bredare allm\"an publik.

Den h\"ar avhandlingen presenterar fem exempel p\aa\ applikationsutveckling inom omr\aa dena finita elementmetodsmodellering, medicin, spanings- och r\"addningstj\"anst samt astronomi.  F\"or finita elementmetodsmodelleringen presenteras ett verktyg och komprimeringsmetod  f\"or att analysera simuleringar av stresstensorer i ett m\"anskligt hj\"arta.  F\"or den medicinska dom\"anen presenteras ett analyssystem som hj\"alper kirurger att styra elektroder vid hj\"arnstimuleringsoperationer.  H\"ar kombineras flera olika modaliteter f\"or att f\"orb\"attra resultatet av operationen.  F\"or spanings- och r\"addningstj\"ansten presenteras ett analyssystem som hj\"alper dem att hitta och r\"adda skadade personer genom ett ett mer sofistikerat beslutst\"odssystem.  Inom astromidom\"anen presenteras f\"orst en utforskningsapplikation f\"or tidsvarierande plasmasimuleringar som anv\"ands f\"or att f\"orb\"attra rymdv\"adersimuleringar.  Slutligen presenteras ett system som fokuserar p\aa\ att kombinera alla tre kategorier i en enda applikation.  Samma verktyg kan d\"armed anv\"andas f\"or utforskning, analys och publik framst\"allning.  Inom astronomidom\"anen m\aa ste ett s\aa dant verktyg kunna hantera koordinatsystem med stor utstr\"ackning och h\"ogkvalitativ \aa tergivning av planetytor samt rymduppdrag.
