\chapter*{Abstract}
\addcontentsline{toc}{chapter}{Abstract}

The dramatic increase in available computational resources over the past two decades has led to an explosion of information that needs to be analyzed.  These increases led to a societal phase transition from a information-poor research environment to being information-rich in which for many problems there is too much information for a single individual to make sense of.  The use of visualization has been important since the beginning of mankind, but this exponential grow of the accessibility of information only further outpaces the relatively constant human capabilities.  Visualization, as a method of transforming complicated data into a representation that is easier for a human to reason about and derive meaning from, can help mitigating this growing discrepancy and thus increase our collective insight.  This is one of the reasons why the importance and the need for visualization research has been  strong in the past and is only ever increasing in the future.

Almost the entirety of visualization research is, directly or indirectly, aimed at the use by a user in an application domain, such as medicine, biology, physics, material sciences, or countless other domains.  Direct research is aimed at the generation of visualization applications that solve a specific problem within an application domain.  By combining prior research and applying it to a concrete problem, it is possible to compare and determine the usability and usefulness of visualization techniques.  These applications can only be effective when the domain experts are closely involved in the design process, leading to an iterative workflow.  These visualization solutions can be separated into three categories:  \emph{Exploration}, in which a domain expert performs an initial study of data, \emph{Production}, in which an established technique is repeatedly applied to a large number of datasets, and \emph{Public Dissemination} in which the findings of domain scientists are spread to a wide audience.

This thesis presents examples of application development in each of these categories that, combined, span a large scale difference and cover a large variety of processes.  While the scales are different between these examples, the underlying principles that inform the design of visualization applications is widely applicable.  For the \emph{Exploration} phase, two applications for plasma physics are presented, one system that enabled researchers to discover the behavior of colliding ion beams; another system that combines simulations and satellite measurements to predict the plasma conditions in the solar system, referred to as space weather.  In the \emph{Production} phase, this work presents applications in biology, where a compression system is presented that enables the real-time rendering of finite element models, work on a system designed to integrate variety of information sources during deep brain stimulation operations, and an application that support rescuers during an urban search \& rescue scenario.  For the \emph{Public Dissemination}, this work shows the ongoing efforts of developing a framework that is used to disseminate astrophysical phenomena to the general public and the efforts that are required to cover these large scales seamlessly.