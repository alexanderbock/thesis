\chapter*{Abstract}
\addcontentsline{toc}{chapter}{Abstract}

Exponential increases in available computational resources over the recent decades have fueled an information explosion in almost every scientific field.  This has led to a societal change shifting from an information-poor research environment to an over-abundance of information.  As many of these cases involve too much information to directly comprehend, visualization proves to be an effective tool to gain insight into these large datasets.  While visualization has been used since the beginning of mankind, its importance is only increasing as the exponential information growth widens the difference between the amount of gathered data and the relatively constant human ability to ingest information.  Visualization, as a methodology and tool of transforming complex data into an intuitive visual representation can leverage the combined computational resources and the human cognitive capabilities in order to mitigate this growing discrepancy.

A large portion of visualization research is, directly or indirectly, targets users in an application domain, such as medicine, biology, physics, or others.  Applied research is aimed at the creation of visualization applications or systems that solve a specific problem within the domain.  Combining prior research and applying it to a concrete problem enables the possibility to compare and determine the usability and usefulness of existing visualization techniques.  These applications can only be effective when the domain experts are closely involved in the design process, leading to an iterative workflow that informs its form and function.  These visualization solutions can be separated into three categories:  \emph{Exploration}, in which users perform an initial study of data, \emph{Production}, in which an established technique is repeatedly applied to a large number of datasets, and \emph{Public Dissemination} in which findings are published to a wider public audience.

This thesis presents five examples of application development in finite element modeling, medicine, urban search \& rescue, and astronomy.  For the finite element modeling, an exploration tool for simulations of the stress tensors in a human heart uses a compression method to achieve interactive frame rates.  In the medical domain, a production system aimed at guiding surgeons during Deep Brain Stimulation interventions fuses multiple modalities in order to improve their outcome.  A second production application is targeted at the Urban Search \& Rescue community to support the extraction of injured victims and enable a more sophisticated decision making strategy.  For the astronomical domain, first, an exploration application enables the analysis of time-varying volumetric plasma simulations in the context of improving the simulations and thus better predict space weather.  A final system focusses on combining all three aspects into a single application that enables the same tools to be used for exploration, production, and public dissemination, thus requiring the handling of large coordinate systems, and high-fidelity rendering of planetary surfaces and spacecraft operations.
