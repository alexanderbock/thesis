\chapter{Visualization Applications}
\label{cha:visapp}

{\textbf Point of the chapter} 
\begin{itemize}
\item "introduces the concepts that i have used in my work"
\item Telling a story that leads up to contribution 
\item Shortest path for a new person to understand the contributions
\item ''This is what people think about collaboration and I did that''
\item ''Supported claims, needs more academic tone''
\item ''Generic, explain the methodolgy''
\end{itemize}


\begin{itemize}
\item Different phases for applications
\item Different kinds of application requirements
\item Some applications might move between phases
\item Some are never designed to move
\item Interplay between Visualization and applications from a scientific (= domain expert) point of view
\item Important aspect for applications: What are the generalizable aspects that other people can use
\item \cite{munzner2009nested} Main paper; four layer model; validation and iterative loops
\item \cite{tory2004human} Different approaches for system construction (design philosophies)
\item \cite{kirby2013visualization} Sci collaborations across disciplines; interdis, multidis, intradis; definition of a domain expert
\item \cite{van2006bridging} kinds of gaps; knowledge gap and interest gap; different cooperation models
\item \cite{wang2000guidelines} Guidelines for multiview setups in information visualization
\item Combining multiple applications to deal with a problem \cite{rungta2013manyvis}
\item ''Participatory design''
\item Different kinds of application requirements: exploration, production, dissemination
\item Learning about the terminology that experts use
\end{itemize}

\section{Exploration Phase}
\begin{itemize}
\item Exploratory for single dataset (information gathering)
\item Initial information gathering
\item Hypothesis forming (first visualization of a new phenomenon)
\end{itemize}

\section{Production Phase}
\begin{itemize}
\item Repeated information gathering (generating tools for looking at the same kind of data over and over)
\item Applying the same techniques to more datasets
\item Repeating processes on multiple datasets
\end{itemize}

\section{Public Dissemination Phase}
\begin{itemize}
\item Presenting information to the general public
\item Public dissemination
\item Robustness
\end{itemize}

\section{Evaluations}
\begin{itemize}
\begin{itemize}
    \item Many papers in InfoVis have been written about evaluations
    \item How are they applicable to scivis or vis in general
\end{itemize}
\item \cite{tory2005evaluating} How to perform expert usability studies
\item \cite{kosara2003thoughts} Different reasons for user studies; alternatives to user studies; hard to publish null results
\item \cite{carpendale2008evaluating} InfoVis evaluation; system v system has bias towards familiar system; applicable to scivis? eval has mixture factors; type 1 v type 2 errors; participatory observation (collaborative work with experts)
\item \cite{lewis1993task} Think aloud protocol introduction to the HCI community
\item \cite{ericsson1980verbal} Variation on the think aloud protocol to only mention actions rather than thoughts
\item \cite{likert1932technique} Introduction of the Likert scale
\item \cite{nielsen1994heuristic} Usability heuristics
\item Evaluations \cite{plaisant2004challenge} (how to report evaluations: \cite{forsell2012guide})
\end{itemize}

\section{Comparison to Software Engineering}
\begin{itemize}
\item Tamara Munzner's nested design is similar to software engineering waterfall (\cite{royce1970managing, victor2003iterative}) model (make a bigger point out of this?)
\item Make comparisons to Scrum + sprits
\end{itemize}

\section{Domain areas}
\label{app:domain}

\begin{itemize}
\item Moving all of the background information from Chapter 4 into this
\item Define challenges that are picked up by the contributions
\end{itemize}
