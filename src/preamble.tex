\documentclass[12pt,a4paper,twoside]{memoir} 
%% draft option for highlighting overfull hboxes and discarding all images
%\documentclass[12pt,a4paper,twoside,draft]{memoir} 

%% to include selected figures in draft mode use
%%    \includegraphics[draft=false]{image.pdf}

\usepackage{silence} % So that we can filter out unwanted warnings and error messages
% http://tex.stackexchange.com/questions/183149/cant-silence-a-pdftex-pdf-inclusion-multiple-pdfs-with-page-group-error
% Not supported on all versions:
% http://tex.stackexchange.com/questions/198586/conditional-based-on-the-version-of-pdflatex
% % supress warning: pdfTeX warning: pdflatex.exe (file <FILENAME>.pdf): PDF inclusion: multiple pdfs ...
\begingroup\expandafter\expandafter\expandafter\endgroup
\expandafter\ifx\csname pdfsuppresswarningpagegroup\endcsname\relax
\else
  \pdfsuppresswarningpagegroup=1\relax
\fi

\usepackage{amsmath,bm}
\usepackage{amssymb}
\usepackage{amsfonts}
\usepackage{amsthm}
%\usepackage[utopia]{mathdesign}
%\DeclareMathAlphabet{\mathcal}{OMS}{cmsy}{m}{n}

\usepackage{fixltx2e}
\usepackage{etoolbox}

\usepackage{lmodern}

\usepackage[square,numbers,sort]{natbib}

\makeatletter
\let\saved@bibitem\@bibitem
\makeatother

%% allows for inclusion of the full bib entry in the text with \bibentry{bibtexkey}
\usepackage{bibentry}
%% tell bibentry to reuse information from the global bibliography
\nobibliography*

%Import usebib package and declare the fields that can be used.
\usepackage{usebib}
\newbibfield{Author} 
\newbibfield{Title} 
\newbibfield{Booktitle} 
\newbibfield{Journal} 
\newbibfield{Publisher} 
%\bibinput{references}

%\usepackage[scaled=.92]{helvet}

\usepackage[svgnames]{xcolor}
%\usepackage{transparent}

\RequirePackage[noflipbooks]{flipbooktheme}

%% Wallpaper for paper dividers
\usepackage{wallpaper}

%% For page dividable boxes
%\usepackage{mdframed}
\usepackage[breakable]{tcolorbox} 
\tcbuselibrary{skins}

%\usepackage[draft]{stgtthesis}
%\usepackage{stgtthesis}
\RequirePackage{liuthesis}

%% google suggested pxfonts instead of mathpazo
%\usepackage{pxfonts}

%% fonts to typeset mathematics to match Palatino, sc = small caps
%\usepackage[sc]{mathpazo}
%\linespread{1.05}         % Palatino needs more leading (space between lines)

%% package for a wide range of symbols, e.g. \textcopyright and \textcircled
\usepackage{textcomp}

%% needed for \ding{<number>} symbols
\usepackage{pifont}

%% MnSymbol automatically loads amsmath, so load amsmath with options before MnSymbol
%\usepackage{amsmath}
%%\usepackage[fleqn]{amsmath}  %% position all equations a fixed distance from left
%% [leqno, reqno] position equations number on the left / right
%\usepackage{amssymb} %% do not use amssym or amsfonts in combination with MnSymbol
%\usepackage{amssymb}
%\usepackage{amsfonts}
%\let\mathdollar\relax
%\usepackage{MnSymbol} % Defines required \filledmedtriangleup
%\let\mathdollar\relax

%% used for theorems
%\usepackage{amsthm}

%\usepackage{MnSymbol,libertine,libertine-math}
% \usepackage{MnSymbol}
% \usepackage{libertine}
% \usepackage{libertine-math}
%%\renewcommand*\oldstylenums[1]{{\fontfamily{fxlj}\selectfont #1}}
%\usefont{T1}{fxlj}{m}{n}\selectfont

%% special handling for ligatures (when copied from the pdf)
%\input{glyphtounicode}
%\pdfglyphtounicode{f_f}{FB00}
%\pdfglyphtounicode{f_f_i}{FB03}
%\pdfglyphtounicode{f_f_l}{FB04}
%\pdfglyphtounicode{f_i}{FB01}
%\pdfgentounicode=1

%% double stroke font for Real an Im stuff I guess
\usepackage{dsfont}

%% takes care of \"{o} vs. ö stuff
\usepackage[T1]{fontenc}

%% enable typesetting with full character protrusion and font expansion
\usepackage{microtype}
%% indicate overfull hboxes with a bar in the margin
\setlength{\overfullrule}{5pt}
%\setlength{\overfullrule}{0pt}

%% ensure that all subscripts are on the same height, even with superscripts
\usepackage{subdepth}

\usepackage{geometry}

\usepackage{enumitem}
\setlist{leftmargin=*,parsep=0pt,itemsep=7pt,topsep=2pt,partopsep=0pt}
\setlist[enumerate]{widest=0}

\usepackage{quoting}
\quotingsetup{vskip=0pt}

\usepackage{tikz}

\usepackage{graphicx}
%%\usepackage{pgfplots}
\usepackage{wrapfig}

%% SI units and number formatting
\usepackage[detect-weight,binary-units]{siunitx}
%% setup si units
\sisetup{list-final-separator={, and }}
%% transfer settings to pgf package
%\SendSettingsToPgf

%% examples:
% \SI[exponent-base=2]{e10}{\kibi\byte} % --> 2^{10}\;KiB
% \sisetup{exponent-to-prefix}
% \SI(5e3}{\g} % --> 5\;kg
% \sisetup{exponent-to-prefix=false}
% \si[prefixes-as-symbols=false]{\kibi} byte % --> 2^{10} byte

\usepackage{units}

%\usepackage[caption=false,farskip=0pt]{subfig}

\usepackage[labelformat=simple]{subcaption}
%% uncomment next line to get subrefs in parenthesis '1a' will become '1(a)'
%% (Note: Since parens is the default label format you will get double parentheses in sub-captions when not specifiying a different label format, e.g. simple.)
\renewcommand\thesubfigure{(\alph{subfigure})}
% Set figure label font. There exist no option when loading the package.
\captionsetup{labelfont={color=niceblue,bf}, textfont=sf,compatibility=false} % sf for sans serif
%% include section in equation numbering
%%\numberwithin{equation}{section}

%% used for accronym table
\usepackage{multicol}

%% allows tables to fill \textwidth with X instead p{4cm}
%\usepackage{tabularx}
%\usepackage{longtable}
%% combined tabularx with longtable
\usepackage{ltxtable}
%\usepackage{ltablex}

\usepackage{colortbl}

%% package for colored to do notes at the margin
\usepackage[colorinlistoftodos,color=green!40]{todonotes}

%% enables re-definitions of macros
\usepackage{letltxmacro}

\usepackage{url}
\usepackage{lipsum}


%% make footnotes appear at the bottom of the page 
%% (and not at the last paragraph)
%\usepackage[bottom]{footmisc}

\usepackage[pdftex,
    pagebackref=true,
    colorlinks,linkcolor=niceblue,
    anchorcolor=niceblue,
    citecolor=niceblue,
    urlcolor=niceblue,
    bookmarks=true,
    bookmarksnumbered=true,
    linktocpage=true,
%   pdfproducer=me,
%   pdfcreator=blub,
    pdfpagemode=UseNone]{hyperref}

% Force linebreaks for doi links in listofpublications.tex
% Cannot use doi package...
%\usepackage{doi}
\DeclareUrlCommand\doi{\def\UrlLeft##1\UrlRight{doi:\href{http://dx.doi.org/##1}{##1}}\urlstyle{rm}}

\pdfminorversion=5

%% Mouse-over pdf tootips!
%\usepackage[createtips]{fancytooltips}
%\usepackage[filename=bibtips,mouseover]{fancytooltips}
%\usepackage{pdfcomment}

%% links to tables, figures, etc. point to the top of the image
%% instead to the caption
%\usepackage[all]{hypcap}


%% provides a list of acronyms
%\usepackage[printonlyused,withpage,smaller]{acronym}
\RequirePackage[smaller]{acronym}
%% makes first letter upper case (used in acronym)
\usepackage{mfirstuc}

%% provides algorithm environment
\RequirePackage[colored]{algorithm2e}
\setlength{\algomargin}{10mm}
%\usepackage{float}
%\newfloat{algorithm}{t}{lop}
%\floatname{algorithm}{Algorithm}

%% provides optional spaces at the end of macros depending on the next character
\usepackage{xspace}


%% provides big braces at the side of a table
%\usepackage{multirow,bigdelim}

%% needed for transparencies in pdfs (imported from inkscape)
%\usepackage{transparent}

%% package for displaying line numbers on the margin
%% (numbering has to be enabled by \linenumbers)
%% print numbers on inner border
\usepackage[switch*]{lineno}
%% print numbers on outer border
%\usepackage[switch]{lineno}

%% load various definitions and commands for the thesis
%\usepackage{thesisdefs}
%\usepackage{thesismathdefs}
%% captions on the side of figures instead of below
\RequirePackage{bysidecap}

%% Include external pdf's
\usepackage{pdfpages}

%% relative font sizes
\usepackage{relsize}

\usepackage{import} % Used to import text-separated pdf not in the root folder

\usepackage{tcolorbox}

\newtoggle{isprint}

%Remove Page numers for parts in TOC
\usepackage{tocloft}
\addtocontents{toc}{\cftpagenumbersoff{part}}
\renewcommand{\cftdot}{}

%% prevent hyperref links on \citeauthor etc.
\makeatletter
\pretocmd{\NAT@citexnum}{\@ifnum{\NAT@ctype>\z@}{\let\NAT@hyper@\relax}{}}{}{}
\makeatother

%% adjust column width of page numbers in Table of Contents
\makeatletter
\renewcommand{\@pnumwidth}{2.55em} %% default 1.55em
\renewcommand{\@tocrmarg}{3.55em}  %% default 2.55em
\makeatother

%%% widow and orphan lines (from memman.pdf, page 50)
% set \clubpenalty, etc. to distinctive values for use
% in tracing page breaks. These values are chosen so that
% no single penalty will absolutely prohibit a page break, but
% certain combinations of two or more will.
\clubpenalty=9996
\widowpenalty=9999
\brokenpenalty=4991
% Reiterate the default value of \redisplaypenalty, for
% completeness.
% Set postdisplaypenalty to a fairly high value to discourage a
% page break between a display and a widow line at the end of a
% paragraph.
\predisplaypenalty=10000
\postdisplaypenalty=1549
% And then \displaywidowpenalty should be at least as high as
% \postdisplaypenalty, otherwise in a situation where two displays
% are separated by two lines, TeX will prefer to break between the
% two lines, rather than before the first line.
\displaywidowpenalty=1602

%\setstocksize{297mm}{210mm}
\settrims{0pt}{0pt}
%% increase the text height slightly
\settypeblocksize{1.05\textheight}{\textwidth}{*} 
%% offset for the binding (shifts the text block outwards)
\setbinding{6mm}
%% ratio between outer and inner margin
\setlrmargins{*}{*}{1.5}
%% ratio between footer and header margin
\setulmargins{*}{*}{1.4}

\checkandfixthelayout